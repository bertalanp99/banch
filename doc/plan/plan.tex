%!TEX TS-program = xelatex
%!TEX encoding = UTF-8 Unicode

\documentclass[a4paper]{article}

%%% PACKAGES %%%
\usepackage{polyglossia}	% Hungarian support
	\setdefaultlanguage{magyar}

\usepackage{graphicx}		% images
	\graphicspath{{../../build/doc/html/}}

\usepackage{color}
\usepackage{listings}		% code listing
	\definecolor{codegreen}{rgb}{0,0.6,0}
	\definecolor{codegray}{rgb}{0.5,0.5,0.5}
	\definecolor{codepurple}{rgb}{0.58,0,0.82}
	\definecolor{backcolour}{rgb}{0.95,0.95,0.92}

	\lstdefinestyle{mystyle}{
		backgroundcolor=\color{backcolour},
		commentstyle=\color{codegreen},
		keywordstyle=\color{magenta},
		numberstyle=\tiny\color{codegray},
		stringstyle=\color{codepurple},
		basicstyle=\footnotesize,
		breakatwhitespace=false,
		breaklines=true,
		captionpos=b,
		keepspaces=true,
		numbers=left,
		numbersep=5pt,
		showspaces=false,
		showstringspaces=false,
		showtabs=false,
		tabsize=4
	}

	\lstset{style=mystyle}

\usepackage{fancyhdr}		% header and footer
	\pagestyle{fancy}
		\fancyhf{} % clear
		\rhead{Péter Bertalan Zoltán (QO7CU6)}
		\lhead{\today}
		\lfoot{\LaTeX}
		\rfoot{\thepage}

%%% TITLE STUFF %%%

\title{\Huge banch \\ \small (Programozás alapjai 2 --- házi feladat terv)}
\date{\today}
\author{Péter Bertalan Zoltán \\ (QO7CU6)}

%%% BODY %%%

\begin{document}
\maketitle

\thispagestyle{fancy}

\section*{Pontosított specifikáció}

A \emph{banch} valójában egy igen egyszerű szoftver. Mindössze arról szól, hogy
van egy \texttt{Recipe} osztály, ezeknek a példányait kezeli a program. Ez az
osztály egy heterogén tárolóban tárolja a különböző hozzávalókat
(\texttt{Ingredient}), igazából egy általam implementált halmaz osztályt használ
erre (\texttt{Set}). Elméletileg egy-egy recept tetszőleges mennyiségű
hozzávalót tartalmazhat.

A felhasználónak van lehetősége \texttt{Recipe} példányokat létrehozni és ezeket
,,megtölteni'' \texttt{Ingredient}-ekkel. Persze az \texttt{Ingredient}
osztálynak vannak leszármazottai, hogy valóban heterogén kollekcióról
beszélhessünk: például vannak alkoholos italok, amiknek van pluszban
alkoholtartalom adattagjuk.

\section*{nostl}

Mivel a feladatban meg volt tiltva, hogy STL tárolókat használjak, ezért
kénytelen voltam megírni a sajátjaimat. Szükségem volt \texttt{String}-re és
\texttt{Set}-re, viszont az utóbbihoz kellett írnom valamilyen okosabb tárolót,
mint a tömb, ezért csináltam egy \texttt{List}-et is. A \texttt{Set}-et (és
ezért a \texttt{List}-et is) úgy írtam meg, hogy tetszőleges típussal működjenek
(\emph{template}). Lentebb egy pár diagram (sajnos a Doxygen által generált
diagramok nem írják a változók típusait; remélem, ez nem túl nagy probléma)

\subsection*{\texttt{nostl::List}}

\begin{center}
\includegraphics[width=3cm, height=5cm]{classnostl_1_1List__coll__graph.png}
\end{center}

Beszúrok ide egy algoritmust az osztályból, hogy legyen valami:

\begin{lstlisting}[language=C++]
template <typename T>
void List<T>::prepend(T const & val)
{
	Node * new_node = new Node;
	new_node->value_ = val;

	// setting new_node's pointers
	new_node->previous_ = this->head_;
	new_node->next_ = this->head_->next_;

	// setting neighbouring nodes' pointers
	new_node->previous_->next_ = new_node;
	new_node->next_->previous_ = new_node;

	// incrementing counter
	++this->number_of_elements_;
}
\end{lstlisting}

(ez az algoritmus felelős azért, hogy a láncolt lista elejére beszúrhassunk
valamit)

\subsection*{\texttt{nostl::Set}}

\begin{center}
\includegraphics[width=2cm, height=3cm]{classnostl_1_1Set__coll__graph.png}
\end{center}

A \texttt{Set} osztály, mint látható, tartalmaz egy \texttt{List} adattagot, és
tulajdonképpen semmit sem csinál, csak egy ,,wrapper'' akörül, annyi bónusszal,
hogy nem enged egy már létező elemet újra a halmazba tenni.

Tehát voltaképpen ez a lényege:

\begin{lstlisting}[language=C++]
template <typename T>
void Set<T>::insert(T const & val)
{
	// make sure list/set doesn't contain item yet
	if (this->size() != 0)
	{
		for (typename List<T>::Iterator i = this->list_.begin();
				i != this->list_.end(); ++i)
		{
			if (*i == val)
			{
				return;
			}
		}
	}

	this->list_.append(val);
}
\end{lstlisting}

\subsection*{\texttt{nostl::String}}

\begin{center}
\includegraphics[width=2cm, height=6cm]{classnostl_1_1String__coll__graph.png}
\end{center}

Ezt az osztályt nem szeretném részletezni, mert tulajdonképpen ugyanaz, amit már
laboron is megírtunk.

\hspace{2cm}
\hrule
\hspace{2cm}

\section*{measure (\texttt{measure::Measure})}

Ez egy kis mini-header fájl tulajdonképpen, ami csak arra hivatott, hogy
értelmesen lehessen kezelni a programban a mennyiségeket.

\begin{center}
\includegraphics[width=2cm, height=6cm]%
					{classmeasure_1_1Measure__coll__graph.png}
\end{center}

\begin{lstlisting}[language=C++]
template <typename T>
class Measure {

	static_assert(std::is_arithmetic<T>::value, "Type must be numeric!");

public:
	Measure(nostl::String const unit, T const & value)
	{
		this->unit_ = unit;
		this->value_ = value;
	}

	bool isSameDimension(Measure const & comparison) const
	{
		return (this->unit_ == comparison.unit_);
	}

	bool operator==(Measure const & rhs) const
	{
		return (isSameDimension(rhs) && (this->value_ == rhs.value_));
	}

	template <typename F>
	friend std::ostream & operator<<(std::ostream &, Measure<F> const);

	// TODO more operators may be needed


private:
	nostl::String unit_;
	T value_;
}; // class Measure
\end{lstlisting}

\hspace{2cm}
\hrule
\hspace{2cm}

\section*{banch}

Végülis itt van a lényeg, itt van a fő projekt. Irónikusan errő tudok a
legkevesebbet mondani, mert a nagy munka az STL tárolók újraírása volt, maga a
recept nyilvántartó program megírása már nem egy olyan bonyolult dolog.

Így néz ki tehát a \texttt{Recipe} osztály diagramja dicső valójában (egyelőre):

\begin{center}
\includegraphics[width=3cm, height=10cm]{classbanch_1_1Recipe__coll__graph.png}
\end{center}

\textit{hogy a diagramon miért nem látszik, hogy az osztálynak van egy
\texttt{ingredients\_} adattagja, amiben a hozzávalók vannak, nem tudom \ldots}

\hspace{2cm}
\hrule
\hspace{2cm}

\section*{Tesztelés}

A szoftvert modulokra bontottam és a Google Test keretrendszerrel tesztelem
folymatosan. Nagyjából így néz ki egy ilyen teszt:

\begin{lstlisting}[language=C++]
TEST(ListTest, Removal)
{
	// create new List with a few elements like {3, 30, 3, 30}
	nostl::List<int> foo;
	foo.append(3);
	foo.append(30);
	foo.append(3);
	foo.append(30);
	EXPECT_EQ(4, foo.size());

	// remove an element
	foo.remove(30); // --> {3, 3, 30}
	nostl::List<int>::Iterator i = foo.begin();
	EXPECT_EQ(3, *(i++));
	EXPECT_EQ(3, *(i++));
	EXPECT_EQ(30, *i);
	EXPECT_EQ(3, foo.size());

	// remove two more just for fun
	foo.remove(30); // --> {3, 3}
	EXPECT_EQ(2, foo.size()); // FIXME this fails
	foo.remove(3); // --> {3}
	i = foo.begin();
	EXPECT_EQ(3, *i);
	EXPECT_EQ(1, foo.size());
}
\end{lstlisting}

\section*{További megjegyzés}

A \texttt{Recipe} osztály tagfüggvényei még megírásra várnak, de látszik, miket
tud majd: hozzávalót bevenni, kivenni, azokat listázni, illetve magát fájlba
írni. Ezek voltak az elvárások a specifikációban.

Megjegyzem, hogy mindazon felül, ami ebben a dokumentumban szerepel, még szükség
lesz egyéb osztályokra is valószínűleg. De szerintem ez a terv leírja a program
lényegét, a terv célját teljesíti.

\end{document}
