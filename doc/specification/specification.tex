%!TEX TS-program = xelatex
%!TEX encoding = UTF-8 Unicode

\documentclass[a4paper,draft]{article}

%%% PACKAGES %%%
\usepackage{polyglossia}	% Hungarian support
	\setdefaultlanguage{magyar}
\usepackage{fancyhdr}		% header and footer
	\pagestyle{fancy}
		\fancyhf{} % clear
		\rhead{Péter Bertalan Zoltán (QO7CU6)}
		\lhead{\today}
		\lfoot{\LaTeX}

%%% TITLE STUFF %%%

\title{\Huge banch \\ \small (Programozás alapjai 2 --- házi feladat specifikáció)}
\date{\today}
\author{Péter Bertalan Zoltán \\ (QO7CU6)}

%%% BODY %%%

\begin{document}
\maketitle

\thispagestyle{fancy}

\noindent \textit{(Ez nem egyéni feladat, az infocpp oldal HF ötletei közül való, így a specifikáció is nagyjából az ott leírtakank felel meg)}

\section*{Összefoglalás}

\textbf{A \emph{banch}} (fantázianév) egy \textbf{italrecept nyilvántartó program}. Arra alkalmas, hogy italrecepteket lehessen vele létrehozni, elmenteni (azaz fájlba írni). Képes továbbá fájlból beolvasni is recepteket, így legújabb ötleteinket barátainkkal is megoszthatjuk, amennyiben ők is \emph{banch}-t használnak.

\section*{Funkciók, lehetőségek}

\begin{itemize}
	\item a \emph{banch} beépítetten ismer pár alapvető hozzávalót
		\begin{itemize}
			\item de ezeken felül lehetőség van újak hozzáadására
		\end{itemize}
	\item lehetőség van receptek létrehozására, interaktív felhasználói felületen
	\item lehetőség van mind a hozzávalók, mind a receptek egyszerű listázására, ,,böngészésére''
	\item receptet és hozzávalót természetesen ugyanúgy el is lehet távolítani, mint ahogy létrehozni
	\item a hozzávalók és receptek tulajdonképpen egy adatbázist alkotnak, amit a \emph{banch} képes fájlba írni (i.e. menteni) illetve akár fájlból beolvasni
\end{itemize}

\section*{Használat}

\begin{itemize}
	\item indításkor a program beolvas egy alapértelmezett fájlt, de futása során lehet az adatbázist ,,bezárni'' és új fájlt betölteni helyette
	\item interaktívan, menüs rendszerben lehetőségünk van a fent említett funckiók elégzésére (hozzávaló / recept hozzáadása / eltávolítása, listázás, etc.)
	\item (ez nem feltétlenül fog megvalósulni, de lehetne olyan lehetőség is, hogy a program a lehetséges hozzávalókból ,,ajánl'' egy keveréket, azaz véletlenszerűen összerak párat)
\end{itemize}
\end{document}
